% https://www.monash.edu/graduate-research/examination/publication
% https://www.monash.edu/rlo/graduate-research-writing/write-the-thesis
% https://www.monash.edu/rlo/graduate-research-writing/write-the-thesis/writing-the-thesis-chapters/structuring-a-long-text
\ack{
\addtocontents{toc}{}  % Add a gap in the Contents, for aesthetics

It is genuinely surprising to me that I am currently writing this section at the end of a long, but rewarding 3 and a half years. I think I know exactly when I was set on this path towards a PhD in Astrophysics. I was in year 10 at Bendigo South East Secondary College attending a VCE information evening deciding what my units would be in year 11. Previous to that night I thought I wanted to teach music. I made the decision that night to do Gen B mathematics, the precursor to Specialist Mathematics. I’m not sure why I chose to do this unit, maybe I thought maths was fun(?) I now realise was a correct intuition on my behalf but as a 15-year-old I had no idea what maths really was, to be honest. That decision to pursue science over the arts, and some amazing physics and maths teachers in years 11 and 12 (shoutout to Mark and Luigi), pushed me to fall in love with physics and astronomy and follow my love to Monash University where I studied a Bachelor of Science and majored in Physics and Astronomy.

Over the next 4 years, I met so many amazing and passionate students, including every member of Monash Advanced Science and Science Scholar Society, who through my competitive streak pushed me to work as hard as possible. I would like to give specific mention of Alex Heger and Amanda Karakas, who let me as an undergraduate student complete projects with them and gave me my first taste of what it means to research. Alex, you also took me on as a student in my Honours year and allowed me to learn and grow in my early academic career and pushed me to achieve the grades required to enter the PhD program at Monash. Thank you.

To my supervisor, Andy: you are one of the smartest people I have ever met. I want to thank you for giving me the opportunity to be your student. Your advice and your expectations have pushed me to become and improve myself as a scientist over the past 3 years. I will never forget the lessons you have taught me. Looking back over some of my early work it is honestly night and day and I wouldn’t have been able to do any of this without you. So, next time the beer is on me.

To Ilya: thank you for always being there to talk about anything. From the impact of Supernovae on binaries to Shakespeare and Russian Poetry you were always ready to talk about anything. I will never forget your advice when I met you in your office once when I was extremely worried about something about my research: “We aren’t curing cancer”. I interpreted this as “it’s supposed to be fun”. I really appreciate the irony of that memory as I, for the life of me, cannot remember what I was worried about and yet your quote is always in my mind while I work.

I would like to give a special shout-out to the Low-mass stellar group. Our meetings every week have made me learn so much: especially how to really communicate my science. I have loved learning about the amazing work that you all do every week.

To all of my many office mates over the past three years, despite not being able to be in the office for nearly two of those years. Thank you for always being down for a chat. Special shoutout to Giulia, Maddy and Avi who have pushed me to be a better scientist and who I am grateful to call my friends.

To my best friends Isaac and Tom. You will always be my rocks. When I am down and stressed out I know I can always turn to you for advice or even for a quick laugh. I appreciate everything that you do for me. Thank you Tom and Kit for letting me move in with you when I had no other options (bloody covid). You helped me get through the long days and I’ll always be so thankful for you. 

I would also like to thank my parents. Mum and Dad thank you for supporting me and giving me a chance to explore and work as hard as I could to complete this. I could not have done this without your support and love. To my brothers Jordan and Riley. I appreciate you being a part of my life and supporting whatever I do.

To my partner Hannah. I’m so glad I met you. Thank you for putting up with the long nights, the partial breakdowns and making me remember what this all means. I love you.

Finally, I have to thank my dog, Spencer. Who, at the time of writing is behind me snoring.


This research was supported by an Australian Government Research Training Program (RTP) Scholarship.


}