% https://www.monash.edu/graduate-research/examination/publication
% https://www.monash.edu/rlo/graduate-research-writing/write-the-thesis
% https://www.monash.edu/rlo/graduate-research-writing/write-the-thesis/writing-the-thesis-chapters/structuring-a-long-text
\abstract{
\addtocontents{toc}{}  % Add a gap in the Contents, for aesthetics


%Much like early astronomers who found that the Sun was in fact spinning our understanding of stellar rotation grows with every observation we make.
%Recent advancements in time-series space-based photometry have allowed us to probe the rotation of many more stars and in some cases their internal rotation profiles.
%Through this, we have identified that our understanding of the evolution of rotation and its effects on our observation of stars is in some cases flawed.
%Through this wealth of data, we can also begin to try to understand the missing physics that understands our misunderstandings.
%This dissertation investigates three disparate problems in stellar rotation.
%Specifically, it demonstrates that the combination of asterseismology and measurement of surface stellar rotation through photometric variability due to stellar spots can constrain the internal rotation profile of stars better than either of these techniques can on their own.
%Following this we show that not accounting for stellar spots, whose expression is directly tied to stellar rotation, introduces bias' to spectroscopically inferred atmospheric parameters.
%Finally, we investigate the nature of the so-called intermediate period gap and propose that the onset of latitudinal differential rotation is likely the cause of the dearth of observation of stars with intermediate surface rotational periods.
%This thesis represents a review of what we know about stellar rotation and provides proposals for improvements to our understanding of the nature of rotation in stars through observationally driven analysis.
\update{Our understanding of stellar rotation has evolved throughout history}, mirroring the discoveries made by early astronomers who revealed the spinning nature of our Sun. 
In recent times, time-series space-based photometry has revolutionised the study of stellar rotation, enabling us to probe the rotational behaviour of numerous stars, including their internal profiles. 
However, these advancements have unveiled discrepancies in our current understanding of \update{in the evolution of rotation throughout the stellar lifetime and the impact rotation has on star observations}.
This dissertation addresses three distinct challenges in the field of stellar rotation. \\

First, we show that independent constraints on the surface rotation rates of subgiants place stronger constraints on the internal rotation profile shape when performing forward modelling given some observed rotational splittings. We discuss the implications of this result on inferring the missing angular momentum transport required to explain the observed core and surface rotation rates of subgiant stars.
We adopt a model of stellar spectra with stellar spots to investigate the effect that spots have on the inference of stellar parameters. We do this by generating a synthetic population of stellar spectra with physically motivated stellar and spot parameters and fitting those spectra with a spotted and non-spotted model of the stellar atmosphere. We show that if stellar spots are not accounted for in models of the stellar atmosphere a mean scatter of 0.03 dex is introduced to recovered metallicity. We argue that because of this result, stellar spots should be accounted for in models of the stellar atmosphere when precise inference of stellar metallicity is required.
Finally, we discuss the intermediate period gap. We argue that there is little evidence to suggest that we do not observe stars within the gap due to a decrease in activity. We, therefore, propose instead that the gap is the result of the onset of equator-fast latitudinal differential rotation onset at the position of the intermediate period gap. Paradoxically, we show that in a synthetic sample of observed rotation periods, this introduces a bias to larger observed rotation periods resulting in the apparent dearth of observations at precise rotation periods.\\

By synthesising a wealth of observational data, this thesis provides a comprehensive review of our current knowledge about stellar rotation while offering proposals to improve our understanding of this fundamental aspect of stars.
}


