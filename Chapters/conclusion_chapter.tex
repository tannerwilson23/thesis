\newcommand\wfirst{\project{WFIRST}}
\newcommand\plato{\project{PLATO}}
\newcommand\haydn{\project{HAYDN}}


 
\chapter {Summary, Conclusions and Future Work}
\label{chap:conc}

\section{Summary}

Rotation is an often overlooked area of stellar astrophysics and astronomy due to its complexity in modelling and observation.
That being said, over the previous decades, our understanding of the evolution of rotation and its impact on stellar evolution has grown dramatically.
With every observation of rotation that we make, we are learning that our simple implementations and parameterisations in stellar evolution codes do not necessarily account for a number of misunderstood mechanisms underlying the evolution of rotating stars.
We are also learning that rotation can have impacts on the observations that we are able to make of stars.
This is the direct result of the growth of the data boom that we find ourselves in due to the sheer number of stars we have precisely photometrically observed over long-period baselines through missions such as \kepler\ \citep{borucki_kepler_2010}, \ktoo\ \citep{howell_k2_2014}, \tess\ \citep{ricker_transiting_2014} and \gaia{} \citep{distefano_gaia_2022} .

Indeed, this thesis does not provide an exhaustive list of every single effect that rotation has on stellar evolution, nor every gap in our knowledge of its effects.
In this work, we have attempted to provide novel ways to improve our understanding of rotation without requiring more data.
The title of this thesis was deliberately chosen as ``Problems in low-mass stellar rotation" - because, while we do work to provide methods that may lead to conclusions about the effects that rotation has on stellar evolution, these are not closed problems.

\section{Conclusions}

This thesis outlines three problems in low-mass stellar rotation and our attempts to address them through novel methods.
In Chapter \ref{chap:subgiant_ast} we first investigated the subgiant angular momentum transport problem.
This is the disparity between the observations of subgiants' core and surface rotation rates and expected core and surface rotation rates from rotating models of stellar evolution.
Those observed core-to-surface rotation rate ratio of low-luminosity subgiants suggest angular momentum transport one to two orders of magnitude greater than the angular momentum transport currently implemented in rotating stellar evolution codes.
Stronger constraints to those low-luminosity subgiants' internal rotation profile shapes would illuminate the excess angular momentum transport mechanism at play.
We have shown in this thesis that, through applying distinct measurements of stellar rotation, specifically here asteroseismology and periodic photometric variability due to stellar spots, to more precisely constrain the internal rotation profile of low-luminosity subgiants.
While we have only applied this method to a single star, KIC 12508433, we believe the method is easily adopted in other studies to more tightly constrain those internal rotation profiles better than either of those measurements of stellar rotation can individually - without requiring more data.

In Chapter \ref{chap:stellar_spots} we investigate the effect that stellar rotation can have on accurate and precise measurements of atmospheric metallicity and chemical abundances.
Accurate measurement of stellar metallicity can be an integral part of our understanding of the universe, from understanding the origin of elements in the universe to galactic archeology to accurate aging of open clusters.
Astrophysicists rely on accurate measurements of atmospheric metallicity to constrain their models of stellar evolution, and astronomers investigate the predictions made by those models.
This is a cyclic process where if astronomers do not adopt accurate models of the stellar atmosphere, inaccuracy compounds.
We created a sample of synthetic spotted stellar spectra with physically motivated stellar parameters to investigate the effect of adopting a non-spotted model of the stellar atmosphere on the accuracy of recovered stellar parameters.
We found that even when adopting a naive model of the effect of stellar spots through a two-temperature model of the stellar atmosphere, stellar spots introduce a mean scatter of 0.03 to the measured metallicity of main-sequence stars.
This effect is non-negligible. 
A spotted model of the stellar atmosphere should be adopted, or the effect that this level of imprecision can have on our inferences of astrophysics should be considered, especially where precise inference of stellar metallicity is required.

Finally, this thesis investigates the intermediate period gap in Chapter \ref{chap:period_gap}.
The intermediate period gap is an unexplained phenomenon manifesting as a dearth of observations of particular stellar surface rotation periods from photometric variability due to stellar spots.
We propose, through a series of data tests in Appendix \ref{apndx:magnetic} that leading theories do not appear to explain the observed features of the gap adequately. 
We propose another cause for the intermediate period gap: the onset of equator-fast latitudinal differential rotation at $R_o \sim 0.45$ and the variability of the distribution of stellar spots (latitudinally) on the surfaces of stars.
We generate a sample of physically motivated stellar rotation periods and calculate their observed rotation periods given multiple relationships between the scale of differential rotation and $R_o$.
Paradoxically, equator-fast differential rotation introduces a bias to larger observed rotation periods.
This, in conjunction with a swift increase in the scale of differential rotation just below $R_o \ < \ 0.45$ creates a dearth of observations resulting in the gap at $R_o \ \sim \ 0.5$, precisely aligning with the intermediate period gap.
While this result requires more thorough investigations into the effect of latitudinal differential rotation on the stellar light curve and, therefore, observed rotational period, this is a novel proposal that we argue explains all of the observational counterparts\footnote{Or lack thereof, in terms of the implied peculiarity.} to the intermediate period gap without the requirement of new, and complex, physics.

\section{Future Work}

\subsection{Applications of our work}

In this work, we have presented representative studies that provide methods for improving our understanding of rotation in stars.
All of these works have further applications for the wider rotational astronomic community.

Applying a surface rotation prior to inference of the internal rotation profiles of low-luminosity subgiants is a novel way to further constrain the internal rotation profiles of stars without requiring more data.
Without this method, see e.g. \citet{ahlborn_asteroseismic_2020} who suggest precise measurement of up to $\ell = 10$ mode rotational splittings are required for inference of the internal rotation profile of post-main-sequence stars using only the rotational splittings, inference of the internal rotation profile of stars will require extremely long baseline observations of a number of subgiants.
While plans are being made for longer-baseline asteroseismic missions \citep{rauer_plato_2014, akeson_wide_2019, miglio_haydn_2021} the length of the missions required for such investigations (on the order of 10s of years) appear unfeasible - given the recent track record of the length of observational missions.
We propose that this method be applied to our observations of post-main-sequence stars with measured rotational splittings.
As we discuss in the conclusion of Chapter \ref{chap:subgiant_ast}, this method can be applied to any star with observed rotational splittings and a measurement of the surface rotation rate.
This can either come in the form of surface rotational period measurements from the active regions of stars, as was adopted in that work and is a by-product of time series-based photometric observation missions, or through spectroscopic \vsini.
The inclination angle and stellar radius, which usually plague the constraints to our measurements of stars, are independently constrained through asteroseismic investigations of the star, making it possible to strongly constrain the surface rotation rates of stars in this way.
In this respect, a good place to start is with the $\sim$30 subgiant stars for which rotational splittings have been observed.
The combination of the constraints to the internal rotation profiles of low-luminosity subgiants, in a population inference sense, may constrain the angular momentum transport mechanism underlying the post-main sequence angular momentum transport problem.

Regarding measuring the effect that stellar spots have on the measurement of the atmospheric metallicity of stars, several applications and extensions to our work could be adopted.
The spectroscopic nature of active regions is a field in its infancy.
More modelling work is required to understand how the fundamental parameters of stars other than the Sun govern their magnetic fields and the properties of stellar spots.
Stellar spots likely have more complex contributions to spectra than the 2-temperature model we have adopted in that work.
Further, the effect of magnetic fields, the properties of stellar spots and their effect on the stellar spectra are unknown.
For some stars, Doppler imaging can indirectly determine the spot contribution to stellar flux.
Investigations into the variance between spotted and unspotted stars, even through observations of the Sun, are required to parameterise the effect of stellar spots on stellar spectra more accurately.

For example, in this work, we have only considered the effect that the cooler ``spot" regions have on the inference of stellar parameters.
Indeed, as well as the cool regions, magnetically active regions can also be brighter or hotter than the ambient temperature of a star in the form of faculae.
The relative effect of faculae is, comparatively, unknown.
Algebraically, in the model we adopt, a star with a large fractional coverage of cool spots is the same as a star with a small fractional coverage of hot faculae.
This suggests that scatter to inferred stellar parameters will be introduced whether you adopt spots or faculae.

The contribution of spots and faculae could be disentangled from the stellar spectra using the method we adopt in this paper.
The cancellation of the contribution to the stellar flux from spots by faculae, and vice versa, is a proposed mechanism underlying the intermediate period gap that we are currently unable to investigate.
However, with a 3-temperature stellar spectra model of the stellar atmosphere, we could determine whether this is true.
We propose an investigation into the feasibility of using the reduced goodness of fit between a single and 3-temperature contribution stellar spectra of the same effective temperature of stars using a single or 3-temperature contribution model of the stellar spectra.

While this method cannot be used to image the stellar spots on the surfaces of stars directly, it can be used to determine the fractional spot coverage and the relative effect that they may have on observed variations to the stellar flux.
An area of concern when using the transit method for observing exoplanets is the degeneracy between transits and stellar spot contributions.
This method could determine the spot contribution to the stellar spectra of stars with suspected exoplanetary transits.
Exoplanet transits do not impact the stellar spectra; they simply reduce the observed flux, but as described in our work, spots do.
A lack of variation in the stellar spectra before and during suspected transits would indicate that the transit is indeed a transit.

Finally, in this work, we investigated the nature of the intermediate period gap that we conclude is created by the onset of latitudinal differential rotation and variances to the latitudinal probability density function of stellar spots.
Given that this is a novel model, we propose a number of follow-up works in this Chapter and will only restate what we consider to be the most important for brevity.
That being said, confirmation of this conclusion using observations is difficult.
Constraining the scale of latitudinal differential rotation of main-sequence stars appears improbable using current photometric methods \citep[See Section 4.3 of][]{aigrain_hare_2015}.
Fourier transforms of spectroscopic line profiles and time series Doppler image maps of the active regions of stars offer a possible avenue to explore in this regard (See Chapter \ref{chap:intro}).
We know at what observed rotation periods stars are below and above the gap, so a good first step would be to determine the scale of differential rotation (and stellar spot distribution) of stars with similar masses and varying observed stellar rotation periods.
Observation of little latitudinal differential rotation below the gap and differential rotation above the gap would provide clear evidence for this being the main mechanism at play.

Furthermore, more investigations into the evolution of magnetic activity regarding rotation are required.
Only recently has the observation that magnetic activity, through the fractional spot coverage of stars, varies as a result of core-envelope recoupling been made.
While our work proposes that the cause of the decreased photometric variability of stars is the decrease in gross rotation rate, followed by an increase in stellar spot coverage from latitudinal differential rotation, further work is required to determine if this is indeed the case.

\subsection{The future of asteroseismology and observation of stellar rotation through photometric modulation}

While the second \kepler{} mission ended five years ago, the data is still being investigated.
A number of stars, especially low-luminosity subgiants, from which we can determine core and surface rotation rates, have yet to be asteroseismically investigated.
We have shown in this work that the internal rotation profile shape of these stars cannot be measured with current measurements of the rotational splittings.
The feasibility of determining the internal rotation profile of these stars without dedicated long-term photometric missions is low.
That being said, the data boom we are currently experiencing regarding photometric data is not slowing.
The \tess{} photometric mission has been collecting short cadence data for over 4 years.
High-resolution observations of stars have been made that will bring with them a new observing field of stars through which we will be able to measure surface rotation periods and rotational splittings, especially within its continuous viewing zones.
With this data, more and more stars have their surface rotation periods \citep{claytor_tess_2023} and rotational splittings determined.
With the accurate measurement of those quantities and the application of our proposed method to place stronger constraints on the internal rotation profile of these stars with a surface rotation rate constraint, a population inference approach appears to answer the post-main sequence angular momentum transport problem.

The future of precise measurement of stellar rotation comes from space-based high-cadence photometric missions.
The Nancy Grace Roman telescope \citep{akeson_wide_2019}, known initially as \wfirst, is a space-based photometric mission that will observe large patches of the sky with a faster cadence than both the \kepler\ and \tess{} missions.
The telescope is expected to launch in 2026, and it will be some time before the baseline is long enough to infer stellar rotation.
Proposals have been made for targeted high-precision asteroseismically focussed missions: PLAnetary Transits and Oscillations of stars (\plato) \citep{rauer_plato_2014}, the High-precision AsteroseismologY of DeNse stellar fields (\haydn) mission \citep{miglio_haydn_2021}.
When, or if, these missions are launched, our understanding of the nature of the rotation of stars will grow.
This will be both from a larger sample of stars with more precise photometric observations and with longer baselines, which will result in more stars with a larger number of and more precise measurements of rotational splittings, as well as a larger sample of stars with photometrically measured surface rotation rates.

As we have discussed within this thesis, there are problems underlying stellar rotation that we still need the answers for, though we are making progress in answering them.
We will not speculate on the next problems in the low-mass stellar rotation. However, with this level of data, we are certainly focussing on 2nd and 3rd-order effects of rotation, which will no doubt improve our understanding of the evolution of rotation in stars and its effects on the observations we make.