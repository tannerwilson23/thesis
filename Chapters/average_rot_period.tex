In this work we have assumed that the rotation periods we adopt from the cluster-tuned two-zone rotational isochrones of \citep{spada_competing_2020}.
We assume in our work that rotation periods output from their models is the the surface equator rotation periods.
It is, however, not clear whether the output rotational periods from their work are equator rotation periods or in fact the average rotation rate of the shell.
The mapping between 1D surface rotation periods and surface rotation profiles is uncertain.
We explore the impact of this assumption on our results in Section \ref{sec:appendix}.
If we adopt the rotational periods output from the rotational isochrones as the average rotation period of the surface rotation profile then the impact of latitudinal differential rotation on the observed rotation period is minimal compared to if we assume the model rotation periods are the equator rotation period.
Assuming again that the stellar spots are uniformly distributed between the equator and a latitude of $60^{\circ}$, the effect on the observed rotational periods is also the inverse of the effect we see if we assume the output rotation period is the equator rotation period: equator-fast differential rotation decreases the observed rotation period and equator-slow differential rotation increases the observed rotation period.
If the 1D model rotation period is the average surface rotation period then latitudinal differential rotation does not qualitatively impact the inferred evolution of rotation from observed rotation periods.
It also does not produce a decrease in the density of observed rotation periods consistent with the intermediate period gap.

It is unclear on first glance which of the two assumptions here is more appropriate to adopt.
We propose that adopting the model surface rotation period as the equator rotation period is more appropriate here.
Indeed, from the perspective of angular momentum conservation, for the angular momentum of a latitudinally flat and latitudinally differentially rotating star to agree the rotation rate of the equator must be greater than the rotation rate of the latitudinally flat rotation profile.
However, we must consider the formation of the differentially rotating surface.
Beginning from a latitudinally flat rotation profile it is unclear how the equator would be spun-up relative to the pole without invoking angular momentum transport against the angular momentum gradient.
It is clear, however, how the poles would be preferentially spun-down relative to the equator.

Let's consider a very simple model of a latitudinally differentially rotating star.
In this model the surface of the star is represented by a series of stacked rings with varying radii dependent on latitude.
Assuming the surface of the star is constant density, those rings will have latitude dependent mass and radii: close to the equator the mass and radius will be much greater than at the pole.
It is clear that the moment of inertia of those disks will be much greater at the equator than at the poles.
Initially we will enforce rigid body rotation, through infinite angular momentum transport between the disks.
If we relax that condition by reducing the angular momentum transport between the rings and allow the star to lose some angular momentum, the poles will have much slower spins than the equator with the same loss of angular momentum.

\citet{granzer_spotted_2003} performed magnetohydrodynamic simulations of rotating low-mass stars in the Pleiades and determined the distribution of stellar spots of stars of various masses and rotation rates (see Figure 3. in their work).
We adapt their findings to suit the work we complete here.
We take the sparse grid of grid of equator rotation rates and stellar masses and measure the maximum and minimum latitudes that stellar spots are expressed for those models.
In this work, we assume that stellar spots are uniformly distributed between the maximum and minimum latitudes that stellar spots are expressed.
While their work explicitly contradicts this assumption, their distributions are decidedly non-uniform, their grid is sparse, and we cannot easily interpolate between distributions of stellar spots.
Furthermore, the stellar spot distributions can be skewed towards the equator and the pole, dependent on the mass and rotation rate of the model - any shape of the distribution that we assume does not adequately reflect the presented stellar spot distributions in their work.